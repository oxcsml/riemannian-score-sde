\section{Preliminaries on stochastic Riemannian geometry}
\label{sec:prel-stoch-riem}

In this section, we recall some basic facts on Riemannian geometry and
stochastic Riemannian geometry.  We follow
\cite{hsu2002stochastic,lee2018introduction,lee2006riemannian} and refer to
\cite{lee2010introduction,lee2013smooth} for a general introduction to
topological and smooth manifolds. Throughout this section $\M$ is a
$d$-dimensional smooth manifold, $\TM$ its tangent bundle and $\TMstar$ it
cotangent bundle. We denote $\rmc^\infty(\M)$ the set of real-valued smooth
functions on $\M$ and $\XM$ the set of vector fields on $\M$.

\subsection{Tensor field, metric, connection and transport}
\label{sec:metr-conn-tens}

\paragraph{Tensor field and Riemannian metric}

For a vector space $V$ let
$\mathrm{T}^{k, \ell}(V) = V^{\otimes k} \otimes (V^\star)^{\otimes \ell}$ with
$k, \ell \in \nset$. For any $k, \ell \in \nset$ we define the space of
$(k,\ell)$-tensors as
$\mathrm{T}^{k,\ell} \M = \sqcup_{p \in \M}
\mathrm{T}^{k,\ell}(\mathrm{T}_p\M)$. Note that
$\Gamma(\M, \mathrm{T}^{0,0}\M) = \mathrm{C}^\infty(\M)$,
$\XM = \Gamma(\M, \mathrm{T}^{1,0} \M)$ and that the space of $1$-form on $\M$
is given by $\Gamma(\M, \mathrm{T}^{0,1} \M)$, where $\Gamma(\M, V(\M))$ is a
section of a vector bundle $V(\M)$ \citep[see][Chapter 10]{lee2013smooth}.  For
any $k \in \nset$, we denote
$\mathrm{T}^{\abs{k}} \M = \sqcup_{j=0}^k \mathrm{T}^{j,k-j} \M$.
% \valentin{maybe talk about pushforward and pullback here ?}
$\M$ is said to be
a Riemannian manifold if there exists $g \in \Gamma(\M, \mathrm{T}^{0,2} \M)$ such that for
any $x \in \M$, $g(x)$ is positive definite. $g$ is called the Riemannian metric
of $\M$. Every smooth manifold can be equipped with a Riemannian metric
\cite[see][Proposition 2.4]{lee2018introduction}. In local coordinates we define
$G = \{g_{i,j}\}_{1 \leq i,j \leq d} = \{g(X_i, X_j)\}_{1 \leq i,j \leq d}$,
where $\{X_i\}_{i=1}^d$ is a basis of the tangent space. In what follows we
consider that $\M$ is equipped with a metric $g$ and for any $X, Y \in \XM$ we
denote $\langle X,Y \rangle_{\M} = g(X,Y)$.

\paragraph{Connection}
A connection $\nabla$ is a mapping which allows one to differentiate vector
fields w.r.t other vector fields. $\nabla$ is a linear map
$\nabla: \ \XM \times \XM \to \XM$. In addition, we assume that
\begin{enumerate*}[label=\roman*)]
\item for any $f \in \rmc^\infty(\M)$, $X, Y \in \XM$, $\nabla_{f X}(Y) = f \nabla_X Y$, 
\item for any $f \in \rmc^\infty(\M)$, $X, Y \in \XM$, $\nabla_{X}(fY) = f \nabla_X Y + X(f) Y$.
\end{enumerate*}
Given a system of local coordinates, the Christoffel symbols
$\{\Gamma_{i,j}^k\}_{1 \leq i,j,k\leq d}$ are given for any
$i,j \in \{1, \dots, d\}$ by
$\nabla_{X_i}X_j = \sum_{k=1}^d \Gamma_{i,j}^k X_k$. We
also define the Levi-Civita connection $\nabla$ by considering the additional
two conditions: 
\begin{enumerate*}[label=\roman*)]
\item $\nabla$ is torsion-free, \ie \ for any $X, Y \in \XM$ we have
  $\nabla_X Y - \nabla_Y X = [X,Y]$, where $[X,Y]$ is the Lie bracket between
  $X$ and $Y$,
\item $\nabla$ is compatible with the metric $g$, \ie \ for any $X,Y,Z \in \XM$,
  $X (\langle Y,Z \rangle_\M) = \langle\nabla_X Y, Z\rangle_\M + \langle Y, \nabla_X Z \rangle_\M$.
\end{enumerate*}
We recall that the Levi-Civita connection is uniquely defined since for any
$X,Y,Z \in \XM$ we have
\begin{align}
  2 \prodM{\nabla_X Y}{Z} &= X(\prodM{Y}{Z}) + Y(\prodM{Z}{X}) - Z(\prodM{X}{Y}) + \prodM{[X,Y]}{Z} - \prodM{[Z,X]}{Y} - \prodM{[Y,Z]}{X}  . 
\end{align}
In this case, we have that the Christoffel symbols are given for any
$i,j,k \in \{1, \dots, d\}$ by
\begin{equation}
  \textstyle{\Gamma_{i,j}^k = (1/2) \sum_{m=1}^d g^{km} (\partial_j g_{m,i} + \partial_i g_{m,j} - \partial_m g_{i,j}) ,}
\end{equation}
where $\{g^{i,j}\}_{1 \leq i,j \leq d} = G^{-1}$. Note that if $\M$ is Euclidean
then for any $i,j,k \in \{1, \dots, d\}$, $\Gamma_{i,j}^k = 0$. We also extend
the connection so that for any $X \in \XM$ and $f \in \rmc^\infty(M)$ we have
$\nabla_X f = X(f)$. In particular, we have that
$\nabla_X f \in \rmc^\infty(\M)$. In addition, we extend the connection such
that for any $\alpha \in \Gamma(\M, \mathrm{T}^{0,1} \M)$, $X,Y \in \XM$ we have
$\nabla_X \alpha (Y) = \alpha(\nabla_X Y) - X(\alpha(Y))$. In particular, we
have that $\nabla_X \alpha \in \Gamma(\M, \mathrm{T}^{1,0} \M)$. Note that for any
$X \in \XM$ and $\alpha, \beta \in \mathrm{T}^{\abs{1}} \M$ we have
$\nabla_X (\alpha \otimes \beta) = \nabla_X \alpha \otimes \beta + \alpha
\otimes \nabla_X \beta$. Similarly, we can define recursively $\nabla_X \alpha$
for any $\alpha \in \Gamma(\M, \mathrm{T}^{k,\ell}\M)$ with $k, \ell \in \nset$. Such an
extension is called a covariant derivative.

\paragraph{Parallel transport, geodesics and exponential mapping} Given a
connection, we can define the notion of parallel transport, which transports
vector fields along a curve. Let $\gamma: \ \ccint{0,1} \to \M$ be a smooth
curve. We define the covariant derivative along the curve $\gamma$ by
$D_{\dot{\gamma}}: \ \Xgamma \to \Xgamma$ similarly to the connection, where
$\Xgamma = \Gamma(\gamma(\ccint{0,1}), \TM)$. In particular if $\dot{\gamma}$
and $X \in \Xgamma$ can be extended to $\XM$ then we define
$D_{\dot{\gamma}}(X) = \nabla_{\dot{\gamma}}X \in \XM$. In what follows, we
denote $D = \nabla$ for simplicity. We say that $X \in \Xgamma$ is parallel to
$\gamma$ if for any $t \in \ccint{0,1}$, $\nabla_{\dot{\gamma}}X(t) = 0$. In
local coordinates, let $X \in \Xgamma$ be given for any $t \in \ccint{0,1}$ by 
$X = \sum_{i=1}^d a_i(t) E_i(t)$ (assuming that $\gamma([0,1])$ is entirely
contained in a local chart), then we have that for any $t \in \ccint{0,1}$ and
$k \in \{1, \dots, d\}$
\begin{equation}
  \label{eq:parallel_transport}
  \textstyle{\dot{a}_k(t) + \sum_{i,j=1}^d \Gamma_{i,j}^k(x(t)) \dot{x}_i(t) a_j(t) = 0  .}
\end{equation}
A curve $\gamma$ on $\M$ is said to be a geodesics if $\dot{\gamma}$ is parallel
to $\gamma$. Using \cref{eq:parallel_transport} we get that
\begin{equation}
  \label{eq:geodesics}
  \textstyle{\ddot{x}_k(t) + \sum_{i,j=1}^d \Gamma_{i,j}^k(x(t)) \dot{x}_i(t) \dot{x}_j(t) = 0  .}
\end{equation}
For more details on geodesics and parallel transport, we refer to \citet[Chapter
4]{lee2018introduction}. Parallel transport will be key to define the frame
bundle and the orthonormal frame bundle in \cref{sec:frame-bundle-orth}. In
addition, we have that parallel transport provides a linear isomorphism between
tangent spaces. Indeed, let $v \in \mathrm{T}_x \M$ and
$\gamma: \ \ccint{0,1} \to \M$ with $\gamma(0) = x$ a smooth curve. Then, there
exists a unique vector field $X^v \in \Xgamma$ such that $X^v(x) = v$ and $X^v$ is
parallel to $\gamma$. For any $t \in \ccint{0,1}$, we denote
$\Gamma_0^t: \mathrm{T}_{x} \M \to \mathrm{T}_{\gamma(t)} \M$ the linear
isomorphism such that $\Gamma_0^t(v) = X^v(\gamma(t))$.

For any $x \in \M$ and $v \in \mathrm{T}_x \M$ we denote
$\gamma^{x,v}: \ \ccint{0,\vareps^{x,v}}$ the geodesics (defined on the maximal
interval $\ccint{0, \vareps^{x,v}}$) on $\M$ such that $\gamma(0) = x$ and
$\dot \gamma(0) = v$. We denote
$\msu^x = \ensembleLigne{v \in \mathrm{T}_x \M}{\vareps^{x,v} \geq 1}$. Note
that $0 \in \msu^x$. For any $x \in \M$, we define the exponential mapping
$\exp_x: \ \msu^x \to \M$ such that for any $v \in \msu^x$,
$\exp_x(v) = \gamma^{x,v}(1)$. If for any $x \in \M$,
$\msu^x = \mathrm{T}_x \M$, the manifold is called \emph{geodesically
  complete}. Note that any connected compact manifold is geodesically
complete. As a consequence we have that there exists a geodesic between any two
points $x, y \in \M$ \cite[see][Lemma 6.18]{lee2018introduction}. For any
$x, y \in \M$, we denote $\mathrm{Geo}_{x,y}$ the sets of geodesics $\gamma$
such that $\gamma(0) = x$ and $\gamma(y) = 1$. For any $x, y \in \M$ we denote
$\Gamma_x^y(\gamma) : \ \mathrm{T}_x \M \to \mathrm{T}_y \M$ the linear
isomorphism such that for any $v \in \mathrm{T}_x \M$,
$\Gamma_x^y(v) = X^v(\gamma(1))$, where $\gamma \in \mathrm{Geo}_{x,y}$. Note
that for any $x \in \M$ there exists $\msv^x \subset \M$ such that
$x \in \msv^x$ and for any $y \in \msv^x$ we have that
$\absLigne{\mathrm{Geo}_{x,y}}=1$.  In this case, we denote
$\Gamma_x^y = \Gamma_x^y(\gamma)$ with $\gamma \in \mathrm{Geo}_{x,y}$.

\paragraph{Orthogonal projection} We will make repeated use of orthonormal
projections on manifolds. Recall that since $\M$ is a closed Riemannian manifold
we can use the Nash embedding theorem \citep{gunther1991isometric}. In the rest
of this paragraph, we assume that $\M$ is a Riemannian submanifold of $\rset^p$
for some $p \in \nset$ such that its metric is induced by the Euclidean
metric. In order to define the projection we introduce
\begin{equation}  
  \mathrm{unpp}(\M) = \ensembleLigne{x \in \rset^d}{\text{there exists a unique $\xi_x$ such that $\normLigne{x - \xi_x} = d(x, \M)$}}  . 
\end{equation}
Let $\mathcal{E}(\M) = \interior(\mathrm{unpp}(\M))$. By \citet[Theorem
1]{leobacher2021existence}, we have that $\M \subset \mathcal{E}(\M)$. We define
$\tilde{p}: \ \mathcal{E}(\M) \to \M$ such that for any $x \in \mathcal{E}(\M)$,
$\tilde{p}(x) = \xi_x$. Using \citet[Theorem 2]{leobacher2021existence}, we have
that $\tilde{p} \in \rmc^\infty(\rset^p, \M)$ and that for any $x \in \M$,
$\tilde{P}(x) = \rmd \tilde{p}(x)$ is the orthogonal projection on
$\mathrm{T}_x\M$. Since $\rset^p$ is normal and $\M$ and
$\mathcal{E}(\M)^\complementary$ are closed, there exists $\msf$ open such that
$\M \subset \msf \subset \mathcal{E}(\M)$. Let
$p \in \rmc^\infty(\rset^p, \rset^p)$ such that for any $x \in \msf$,
$p(x) = \tilde{p}(x)$ (given by Whitney extension theorem for
instance). Finally, we define $P: \ \rset^p \to \rset^p$ such that for any
$x \in \rset^p$, $P(x) = \rmd p(x)$. Note that for any $x \in \M$, $P(x)$ is the
orthogonal projection $\mathrm{T}_x \M$ and that
$P \in \rmc^\infty(\rset^p, \rset^p)$.


\subsection{Stochastic Differential Equations on manifolds}
\label{sec:stoch-diff-equat}


\paragraph{Stratanovitch integral} For reasons that will become clear in the
next paragraph it is easier to define Stochastic Differential Equations (SDEs)
on manifolds w.r.t the Stratanovitch integral \cite[Part II, Chapter
3]{kloeden:platen:2011}. We consider a filtered probability space
$(\Omega, (\mcf_t)_{t \geq 0}, \Pbb)$. Let $(\bfX_t)_{t \geq 0}$ and
$(\bfY_t)_{t \geq 0}$ be two real continuous semimartingales. We define the
quadratic covariation $([\bfX,\bfY]_t)_{t \geq 0}$ such that for any $t \geq 0$
\begin{equation}
  \textstyle{[\bfX,\bfY]_t = \bfX_t \bfY_t - \bfX_0\bfY_0 - \int_0^t \bfX_s \rmd \bfY_s - \int_0^t \bfY_s \rmd \bfX_s  . }
\end{equation}
We refer to \citet[Chapter IV]{revuz1999continuous} for more details on
semimartingales and quadratic variations. We denote $[\bfX] = [\bfX, \bfX]$. In
particular, we have that $([\bfX, \bfY]_t)_{t \geq 0}$ is an adapted continuous
process with finite-variation and therefore $[[\bfX, \bfY]] = 0$. Let
$(\bfX_t)_{t \geq 0}$ and $(\bfY_t)_{t \geq 0}$ be two real continuous
semimartingales, then we define the Stratanovitch integral as follows for any
$t \geq 0$
\begin{equation}
  \textstyle{ \int_0^t \bfX_s \circ \rmd \bfY_s = \int_0^t \bfX_s \rmd \bfY_s + (1/2) [\bfX, \bfY]_t  . }
\end{equation}
In particular, denoting $(\bfZ_t^1)_{t \geq 0}$ and $(\bfZ_t^2)_{t \geq 0}$ the
processes such that for any $t \geq 0$,
$\bfZ_t^1 = \int_0^t \bfX_s \circ \rmd \bfY_s$ and
$\bfZ_t^2 = \int_0^t \bfX_s \rmd \bfY_s$, we have that $[\bfZ^1] = [\bfZ^2]$. We
refer to \cite{kurtz1995stratonovich} for more details on Stratanovitch
integrals. Note that if for any $t \geq 0$,
$\bfX_t = \int_0^t f(\bfX_s) \circ \rmd \bfY_s$ with $\rmc^1(\rset, \rset)$,
then $[\bfX, \bfY]_t = \int_0^t f(\bfX_s) f'(\bfX_s) \rmd \bfY_s$. Assuming that
$f \in \rmc^3(\rset, \rset)$ we have that \cite[Chapter IV, Exercise
3.15]{revuz1999continuous}
\begin{equation}
  \label{eq:stratanovitch_lemma}
  \textstyle{ f(\bfX_t) = f(\bfX_0) + \int_0^t f'(\bfX_s) \circ \rmd \bfX_s  .}
\end{equation}
The proof relies on the fact that for any $t \geq 0$,
$\rmd [\bfX, f'(\bfX)]_t = f''(\bfX_t) \rmd [\bfX]_t$.  This result should be
compared with It\^o's lemma. In particular, Stratanovitch calculus satisfies the
ordinary chain rule making it a useful tool in differential geometry which
makes a heavy use of diffeomorphism.

\paragraph{SDEs on manifolds}
We define semimartingales and SDEs on manifold through the lens of their actions
on functions. A continuous $\M$-valued stochastic process $(\bfX_t)_{t \geq 0}$
is called a $\M$-valued semimartingale if for any $f \in \rmc^\infty(\M)$ we
have that $(f(\bfX_t))_{t \geq 0}$ is a real valued semimartingale. Let
$\ell \in \nset$, $V^{1:\ell} = \{ V_i\}_{i=1}^\ell \in \XM^\ell$ and
$Z^{1:\ell} = \{Z^i\}_{i=1}^\ell$ a collection of $\ell$ real-valued
semimartingales. A $\M$-valued semimartingale $(\bfX_t)_{t \geq 0}$ is said to
be the solution of $\SDE(V^{1:\ell}, Z^{1:\ell}, \bfX_0)$ up to a stopping
$\tau$ with $\bfX_0$ a $\M$-valued random variable if for all
$f \in \rmc^\infty(\M)$ and $t \in \ccint{0, \tau}$ we have 
\begin{equation}
  \textstyle{f(\bfX_t) = f(\bfX_0) + \sum_{i=1}^\ell \int_0^t V_i(f)(\bfX_s) \circ \rmd \bfZ^i_s  . } 
\end{equation}
Since the previous SDE is defined w.r.t the Stratanovitch integral we have that
if $(\bfX_t)_{t \geq 0}$ is a solution of $\SDE(V^{1:\ell}, Z^{1:\ell}, \bfX_0)$
and $\Phibf: \M \to \mathcal{N}$ is a diffeomorphism then $(\Phibf(\bfX_t))_{t \geq 0}$
is a solution of $\SDE(\Phibf_\star V^{1:\ell}, Z^{1:\ell}, \Phibf(\bfX_0))$,
where $\Phibf_\star$ is the pushforward operation \cite[see][Proposition
1.2.4]{hsu2002stochastic}. Because the vector fields $\{V_i\}_{i=1}^\ell$ are
smooth we have that for any $\ell \in \nset$,
$V^{1:\ell} = \{ V_i\}_{i=1}^\ell \in \XM^\ell$ and
$Z^{1:\ell} = \{Z^i\}_{i=1}^\ell$ a collection of $\ell$ real-valued
semimartingales, there exists a unique solution to
$\SDE(V^{1:\ell}, Z^{1:\ell}, \bfX_0)$ \cite[see][Theorem
1.2.9]{hsu2002stochastic}.


\subsection{Frame bundle and orthonormal frame bundle}
\label{sec:frame-bundle-orth}

We now introduce the concepts of frame bundle and orthonormal bundle over the
manifold $\M$. These concepts are useful to define stochastic processes on $\M$
using Euclidean stochastic processes. In particular, we will see that a Brownian
motion on the manifold can be linked to the Euclidean Brownian motion using the
orthonormal bundle. For any $x \in \M$, a frame at $x$ is an isomorphism
$f: \ \rset^d \to \mathrm{T}_x \M$. Note that $f$ is equivalent to the choice of
a basis in $\mathrm{T}_x \M$. We denote $\mathrm{F}_x \M$ the set of frames at
$p$. The frame bundle denoted $\FM$ is given by
$\FM = \sqcup_{x \in \M} \mathrm{F}_x \M$. The frame bundle can be given a
smooth structure and is therefore a $d + d^2$-dimensional manifold. Similarly,
for any $x \in \M$, an orthonormal frame at $x$ is a linear isometry
$f: \ \rset^d \to \mathrm{T}_x \M$. Note that $f$ is equivalent to the choice of
an orthonormal basis in $\mathrm{T}_x \M$. We denote $\mathrm{O}_p \M$ the set
of orthonormal frames at $p$. The orthonormal frame bundle denoted $\OM$ is
given by $\OM = \sqcup_{x \in \M} \mathrm{O}_x \M$. The orthonormal frame bundle
can be given a smooth structure and is therefore a $d + d(d-1)/2$-dimensional
manifold. We denote $\pi: \ \FM \to \M$ the smooth projection such that for any
$u = (x,f) \in \FM$, $\pi(u) = x$. Note that the restriction of $\pi$ to the
orthonormal bundle is also smooth.  Frame bundles and orthonormal bundles are
primary examples of principal bundles and we refer to \cite{kolar2013natural}
for more details.

One key element of frame bundles and orthonormal bundles is their link with the
connections on $\M$. Let $u = (x, f) \in \FM$ and
$U \in \mathrm{T}_u \mathrm{F}M$. $U$ is said to be vertical if there exists a
smooth curve $u: \ \ccint{0,1} \to \FM$ such that for any $t \in \ccint{0,1}$,
$\pi(u(t)) = x$ and $\dot u(0) = U$. We say that $U$ is tangent to the fibre
$\mathrm{F}_{\pi(u)}\M$. The space of vertical tangent vectors is called the
vertical space and is denoted $\mathrm{V}_u \FM$. We have that
$\mathrm{dim}(\mathrm{V}_u \mathrm{F}\M) = d^2$. We now define the horizontal
space as follows. Let $u: \ \ccint{0,1} \to \FM$ be a smooth curve. We say that
$u = (f,x)$ is horizontal if for any $t \in \ccint{0,1}$ and
$i \in \{1, \dots, d\}$, $\nabla_{\dot x} (f e_i)(t) = 0$, where
$\{e_i\}_{i=1}^d$ is the canonical basis of $\rset^d$. In other words, the
horizontal curve corresponds to the parallel transport of a frame along a smooth
curve in $\M$. Let $u = (x, f) \in \FM$ and $U \in \mathrm{T}_u
\mathrm{F}M$. $U$ is said to be horizontal if there exists a smooth horizontal
curve $u: \ \ccint{0,1} \to \FM$ such that $\dot u(0) = U$. The space of
horizontal tangent vectors is called the horizontal space and is denoted
$\mathrm{H}_u \FM$. Let $v \in \rset^d$, we define the vector field
$H_v \in \mathcal{X}(\FM)$ such that for any $u \in \FM$, $H_v(u) = \dot u(0)$
with $\gamma=(x,f): \ \ccint{0,1} \to \FM$ a smooth curve on $\FM$ such that
$\dot x(0) = e(0)v$ and $\gamma(0) = u$. The existence of $H_v$ for any
$v \in \rset^d$ is discussed in \citet[p.69-70]{kobayashi1963foundations} and
$H_v$ is called the horizontal lift of $v$. For any $i \in \{1, \dots, d\}$ we
denote $H_i = H_{e_i}$ where $\{e_i\}_{i=1}^d$ is the canonical basis of
$\rset^d$. In particular, since any horizontal curve is entirely specified by
$\gamma(0) = (x(0), f(0))$ and $\dot{x}(0)$, we get that
$\mathrm{dim}(\mathrm{H}_u \FM) = d$ for any $u \in \FM$.

Consider a connection $\nabla$ on $\M$. Note that for any $u = (x,f) \in \FM$, we have
$\mathrm{T}_u \FM = \mathrm{T}_u \M \oplus \mathrm{V}_u \FM$. In local
coordinates $\{x_i\}_{i=1}^d$, we denote $\{X_i\}_{i=1}^d$ a basis of
$\mathrm{T}_x \M$. For any $j \in \{1, \dots, j\}$, there exist
$\{f_{i,j}\}_{i=1}^d$ such that $f e_j = \sum_{i=1}^d f_{i,j} X_i$ (note that
$\{f_{i,j}\}_{1 \leq i,j \leq d}$ can be interpreted as the matrix transforming
a vector of $\rset^d$ into a vector of $\mathrm{T}_x\M$ expressed in the basis
$\{X_i\}_{i=1}^d$). In particular, we have that
$\{x_k, f_{i,j}\}_{1 \leq i,j, k \leq d}$ are local coordinates for $\FM$. We
denote by $\{X_k, X_{i,j}\}_{1 \leq i,j,k \leq d}$ the associated basis in
$\mathrm{T}_u \FM$ for any $u \in \msu$, where $\msu$ is an open subset of $\FM$
on which the local coordinates are well-defined. Leveraging properties of
parallel transport, we have that for any $j \in \{1, \dots, d\}$ and $u \in \msu$
\begin{equation}
  \label{eq:horizontal_lift}
  \textstyle{ H_j(u) = \sum_{i=1}^d f_{i,j} X_i - \sum_{\ell, m=1}^d \{ \sum_{i, k=1}^d f_{i,j} f_{k,m} \Gamma_{i,k}^\ell\} X_{\ell,m}  ,}
\end{equation}
where we recall that $\{\Gamma_{i,j}^k\}_{1 \leq i,j,k \leq d}$ are the
Christoffel symbols of the connection in local coordinates.  In particular, it
is clear that for any $u \in \FM$, $\{H_i(u)\}_{i=1}^d$ is a basis of
$\mathrm{H}_u \FM$ and that $\mathrm{H}_u \FM \cap \mathrm{V}_u \FM = \{0\}$,
hence $\mathrm{T}_u \FM = \mathrm{H}_u \FM \oplus \mathrm{V}_u \FM$. Using
\cref{eq:horizontal_lift} we have that the horizontal space is entirely defined
by the connection $\nabla$. Reciprocally, any smooth linear complement of the
vertical space gives rise to a connection \cite[see][Section
11.11]{kolar2013natural}.

We now illustrate how we can go from a smooth curve on $\M$ (equipped with a
connection $\nabla$) to a smooth curve on $\rset^d$. First, let
$x: \ \ccint{0,1} \to \M$ be a smooth curve on manifold. Define
$f(0) \in \mathrm{F}_{x(0)} \M$ and consider $u: \ \ccint{0,1} \to \FM$ the
smooth horizontal curve associated with $x$ and starting frame $f(0)$. Now
consider the antidevelopment of $u$ given by the smooth curve
$z: \ \ccint{0,1} \to \rset^d$ such that for any $t \in \ccint{0,t}$
\begin{equation}
  \textstyle{ z(t) = \int_0^t f(s)^{-1} \dot x(s) \rmd s   . }
\end{equation}
We now show how a smooth curve on $\rset^d$ gives rise to a smooth curve in
$\M$. First, note that for any $t \in \ccint{0,1}$, we have that
$\dot u (t) = \sum_{i=1}^d H_i(u(t)) \dot z_i(t)$. Hence, specifying $u(0)$ any
smooth curve $z$ on $\rset^d$ is associated to a smooth curve on $\FM$. We
obtain a smooth curve on $\M$ by considering $x = \pi(u)$. In the next section,
we present similar ideas when smooth curves are replaced by semimartingales.

\subsection{Horizontal lift and stochastic development}
\label{sec:horiz-lift-stoch}

We are now ready to present the notion of horizontal semimartingale, which is
key to draw the link between semimartingales on $\M$ and semimartingales on
$\rset^d$. We follow the presentation of \citet[Section
2.3]{hsu2002stochastic}. Again, we consider a filtered probability space
$(\Omega, (\mcf_t)_{t \geq 0}, \Pbb)$. All the semimartingales we consider are
defined w.r.t this filtered probability space. We assume that the manifold $\M$
is equipped with a connection $\nabla$.

\begin{definition}[Stochastic development]
  Let $(\bfZ^{1:d}_t)_{t \geq 0} = \{(\bfZ_t^i)_{t \geq 0}\}_{i=1}^d$ be a
  collection of real-valued semimartingales.  Let $(\bfU_t)_{t \geq 0}$ be the
  $\FM$ semimartingale solution of $\SDE(H^{1:d}, \bfZ^{1:d}, \bfU_0)$ with
  $H^{1:d} = \{H_i\}_{i=1}^d$. $(\bfU_t)_{t \geq 0}$ is called the \emph{stochastic
    development} of $\bfZ^{1:d}$ on $\FM$. Similarly, the $\M$-valued
  semimartingale $(\bfX_t)_{t \geq 0} = (\pi(\bfU_t))_{t \geq 0}$ is called the
  \emph{stochastic development} of $\bfZ^{1:d}$ on $\M$.
\end{definition}

The previous definition allows to transfer a semimartingale on $\rset^d$ to a
semimartingale on $\M$ in an \emph{intrinsic} manner. Reciprocally, we also aim
at transferring a semimartingale on $\M$ to a semimartingale on $\rset^d$.

\begin{definition}[Horizontal lift and antivelopment]
  Let $(\bfX_t)_{t \geq 0}$ be a $\M$-valued semimartingale. If there exist a
  $\FM$-valued semimartingale $(\bfU_t)_{t \geq 0}$ and
  $(\bfZ^{1:d}_t)_{t \geq 0} = \{(\bfZ_t^i)_{t \geq 0}\}_{i=1}^d$ a collection
  of real-valued semimartingales such that
  $(\bfX_t)_{t \geq 0} = (\pi(\bfU_t))_{t \geq 0}$ and $(\bfU_t)_{t \geq 0}$ is
  solution of $\SDE(H^{1:d}, \bfZ^{1:d}, \bfU_0)$ with
  $H^{1:d} = \{H_i\}_{i=1}^d$ then $(\bfU_t)_{t \geq 0}$ is called the
  \emph{horizontal lift} of $(\bfX_t)_{t \geq 0}$ and
  $(\bfZ^{1:d}_t)_{t \geq 0}$ the \emph{antidevelopment} of
  $(\bfX_t)_{t \geq 0}$.
\end{definition}

The existence of an horizontal lift and an antidevelopment is not
trivial. Considering the Nash embedding theorem 
\citep[see for example][]{gunther1991isometric}, it is possible to show the existence
and uniqueness of these processes (up to initialization). Without loss of
generality, we can then assume that $\M \subset \rset^p$ and for any $x \in \M$,
$\mathrm{T}_x \M \subset \rset^p$ with $p \geq d(d+1)/2$ (and
$p \leq \max(d(d+5)/2, d(d+3)/2+5)$). For any $x \in \M$, we denote
$P(x): \ \rset^p \to \mathrm{T}_x \M$ the projection operator. In addition for
any $x \in \M$, we denote $\{P_i(x)\}_{i=1}^p = \{P(x) e_i\}_{i=1}^p$, where
$\{e_i\}_{i=1}^p$ is the canonical basis of $\rset^p$. Note that
$\{P_i\}_{i=1}^p \in \XM^p$. In addition for any $x \in \M$ we denote
$\{x^i\}_{i=1}^p$ its coordinates in $\rset^p$, \ie \ for any
$i \in \{1, \dots, p\}$, $x^i = \langle x, e_i \rangle$. In particular, if
$(\bfX_t)_{t \geq 0}$ is a $\M$-valued process then for any
$i \in \{1, \dots, p\}$,
$(\bfX_t^i)_{t \geq 0} = (\langle \bfX_t, e_i \rangle)_{t \geq 0}$ is a
real-valued process. If $(\bfX_t)_{t \geq 0}$ is a $\M$-valued semimartingale
then it is the solution of $\SDE(\{P_i\}_{i=1}^p, \{\bfX^i\}_{i=1}^p, \bfX_0)$
\cite[see][Lemma 2.3.3]{hsu2002stochastic}. Then, a candidate for the horizontal
lift of $(\bfX_t)_{t \geq 0}$ is given by
$(\bfU_t)_{t \geq 0}=(\bfX_t, \bfE_t)_{t \geq 0}$ solution of
$\SDE(\{P_i^\star\}_{i=1}^p, \{\bfX^i\}_{i=1}^p, \bfU_0)$, where for any
$i \in \{1,\dots,p\}$, $P_i^\star(u) = H_{f^{-1}P_i(\pi(u))}(u)$ and
$\bfX_0 = \pi(\bfU_0)$. We have that $(\bfU_t)_{t \geq 0}$ is the stochastic
development of $\{(\bfZ_t^i)_{t \geq 0}\}_{i=1}^d$ where for any $t \geq 0$,
$\bfZ_t = \sum_{i=1}^p \int_0^t \bfE_s^{-1} P_i(\bfX_s) \circ \rmd \bfX_s^i$
 \cite[see][Theorem 2.3.4]{hsu2002stochastic}. Finally, we have that given
$\bfU_0$, $(\bfU_t)_{t \geq 0}$ is the unique horizontal lift of
$(\bfX_t)_{t \geq 0}$ and $(\bfZ_t)_{t \geq 0}$ is the unique antidevelopment of
$(\bfX_t)_{t \geq 0}$ \cite[see][Theorem 2.3.5]{hsu2002stochastic}.

\subsection{Brownian motion on manifolds}
\label{sec:brown-moti-manif}

In this section, we introduce the notion of Brownian motion on manifolds. We
derive some of its basic convergence properties and provide alternative
definitions (stochastic development, isometric embedding, random walk
limit). These alternative definitions are the basis for our alternative
methodologies to sample from the time-reversal. To simplify our discussion, we
assume that $\M$ is a connected compact Riemannian manifold equipped with the
Levi-Civita connection $\nabla$. We denote $p_{\textup{ref}}b$ the Haussdorff measure of
the manifold (which coincides with the measure associated with the Riemannian
volume form \citep[see][Theorem 2.10.10]{federer2014geometric} and
$p_{\textup{ref}} = p_{\textup{ref}}b / p_{\textup{ref}}(\M)$ the associated probability measure.

\paragraph{Gradient, divergence and Laplace operators}
Let $f \in \rmc^{\infty}(\M)$. We define $\nabla f \in \XM$ such that for any
$X \in \XM$ we have $\langle X, \nabla f \rangle_{\M} = X(f)$. Let
$\{X_i\}_{i=1}^d \in \XM^d$ such that for any $x \in \M$, $\{X_i(x)\}_{i=1}^d$
is an orthonormal basis of $\mathrm{T}_x \M$. Then, we define
$\dive: \ \XM \to \rmc^\infty(\M)$ (linear) 
such that for any $X \in \XM$,
$\dive(X) = \sum_{i=1}^d \prodM{\nabla_{X_i}X}{X_i}$. The following Stokes
formula (also called divergence theorem, see \citet[p.51]{lee2018introduction})
holds for any $f \in \rmc^\infty(\M)$ and $X \in \XM$,
$\int_{M} \dive(X)(x) f(x) \rmd p_{\textup{ref}}(x) = - \int_M X(f)(x) \rmd
p_{\textup{ref}}(x)$. Let $X = \sum_{i=1}^d a_i X_i$ in local coordinates.  Using the
Stokes formula and the definition of the gradient we get that in local
coordinates
\begin{equation}
\textstyle{  \nabla f = \sum_{i,j=1}^d g^{i,j} \partial_i f X_j  ,  \qquad \dive(X) = \det(G)^{-1/2} \sum_{i=1}^d \partial_i(\det(G)^{1/2} a_i)  . }
\end{equation}
The Laplace-Beltrami operator is given by 
$\Delta_{\M} : \ \rmc^\infty(M) \to \rmc^\infty(M)$ and for any
$f \in \rmc^\infty(M)$ by $\Delta_{\M}(f) = \dive(\grad(f))$. In local
coordinates we obtain 
$\Delta_{\M}(f) = \det(G)^{-1/2} \sum_{i=1}^d \partial_i (\det(G)^{1/2}
\sum_{j=1}^d g^{i,j} \partial_j f)$. Using the Nash isometric embedding theorem
\citep{gunther1991isometric} we will see that $\Delta_{\M}$ can always be
written as a sum of squared operators. However, this result requires an
\emph{extrinsic} point of view as it relies on the existence of projection
operators. In contrast, if we consider the orthonormal bundle $\OM$ we can
define the Laplace-Bochner operator
$\Delta_{\OM}: \ \rmc^\infty(\OM) \to \rmc^\infty(\OM)$ as
$\Delta_{\OM} = \sum_{i=1}^d H_i^2$, where we recall that for any
$i \in \{1, \dots, d\}$, $H_i$ is the horizontal lift of $e_i$. In this case,
$\Delta_{\OM}$ is a sum of squared operators and we have that for any
$f \in \rmc^\infty(\M)$, $\Delta_{\OM}(f \circ \pi) = \Delta_{\M}(f)$
\cite[see][Proposition 3.1.2]{hsu2002stochastic}. Being able to express the various
Laplace operators as a sum of squared operators is key to express the associated
diffusion process as the solution of an SDE.

\paragraph{Alternatives definitions of Brownian motion}

We are now ready to define a Brownian motion on the manifold $\M$. Using the
Laplace-Beltrami operator, we can introduce the Brownian motion through the lens
of diffusion processes.

\begin{definition}[Brownian motion]
  Let $(\bfB_t^\M)_{t \geq 0}$ be a $\M$-valued semimartingale.
  $(\bfB_t^\M)_{t \geq 0}$ is a Brownian motion on $\M$ if for any
  $f \in \rmc^\infty(\M)$, $(\bfM_t^f)_{t \geq 0}$ is a local martingale where
  for any $t \geq 0$
  \begin{equation}
    \textstyle{\bfM_t^f = f(\bfB_t^\M) - f(\bfB_0^\M) - (1/2)\int_0^t \Delta_{\M}f(\bfB_s^\M) \rmd s  .}
  \end{equation}
\end{definition}

Note that this definition is in accordance with the definition of the Brownian
motion as a diffusion process in the Euclidean space $\rset^d$, since in this
case $\Delta_{\M} = \Delta$. As emphasized in the previous section any
semimartingale on $\M$ can be associated to a process on $\FM$ (or $\OM$) and a
process on $\rset^d$. The proof of the following result can be found in
\citet[Propositions 3.2.1 and 3.2.2]{hsu2002stochastic}.

\begin{proposition}[Intrinsic view of Brownian motion]
  \label{prop:intrinsic_brownian}
  Let $(\bfB_t^\M)_{t \geq 0}$ be a $\M$-valued semimartingales. Then
  $(\bfB_t^\M)_{t \geq 0}$ is a Brownian motion on $\M$ if and only on the
  following conditions hold:
  \begin{enumerate}[label=\alph*)]
  \item The horizontal lift $(\bfU_t)_{t \geq 0}$ is a $\Delta_{\OM}/2$
    diffusion process, \ie \ for any $f \in \rmc^\infty(\OM)$, we have that
    $(\bfM_t^f)_{t \geq 0}$ is a local martingale where for any $t \geq 0$
  \begin{equation}
    \textstyle{\bfM_t^f = f(\bfU_t) - f(\bfU_0) - (1/2)\int_0^t \Delta_{\OM}f(\bfU_s) \rmd s  .}
  \end{equation}    
\item The stochastic antidevelopment of $(\bfB_t^\M)_{t \geq 0}$ is a
  $\rset^d$-valued Brownian motion $(\bfB_t)_{t \geq 0}$.
  \end{enumerate}
\end{proposition}

In particular the previous proposition provides us with an \emph{intrisic} way
to sample the Brownian motion on $\M$ with initial condition $\bfB_0^\M$. First
sample $(\bfU_t)_{t \geq 0}$ solution of $\SDE(H^{1:d}, \bfB^{1:d}, \bfU_0)$
with $H^{1:d} = \{H_i\}_{i=1}^d$ and $\pi(\bfU_0) = \bfB_0^\M$ and $\bfB^{1:d}$ the
Euclidean $d$-dimensional Brownian motion. Then, we recover the $\M$-valued
Brownian motion $(\bfB_t^\M)_{t \geq 0}$ upon letting
$(\bfB_t^\M)_{t \geq 0} = (\pi(\bfU_t))_{t \geq 0}$.
% habermann

We now consider an \emph{extrinsic} approach to the sampling of Brownian motions
on $\M$. Using the Nash embedding theorem \citep{gunther1991isometric}, there
exists $p \in \nset$ such that without loss of generality we can assume that
$\M \subset \rset^p$. For any $x \in \M$, we denote
$P(x): \ \rset^p \to \mathrm{T}_x \M$ the projection operator. In addition for
any $x \in \M$, we denote $\{P_i(x)\}_{i=1}^p = \{P(x) e_i\}_{i=1}^p$, where
$\{e_i\}_{i=1}^p$ is the canonical basis of $\rset^p$. For any
$i \in \{1, \dots, p\}$, we smoothly extend $P_i$ to $\rset^p$. In this case, we
have the following proposition \cite[Theorem 3.1.4]{hsu2002stochastic}:

\begin{proposition}[Extrinsic view of Brownian motion]
  \label{prop:extrinsic_brownian}
  For any $f \in \rmc^{\infty}(\M)$ we have that
  $\Delta_M(f) = \sum_{i=1}^p P_i(P_i(f))$. Hence, we have that
  $(\bfB_t^\M)_{t \geq 0}$ solution of
  $\SDE(\{P_i\}_{i=1}^{p}, \bfB^{1:p}, \bfB_0^\M)$ with $\bfB_0^\M$ a $\M$-valued
  random variable and $\bfB^{1:p}$ a $\rset^p$-valued Brownian motion.
\end{proposition}

The second part of this proposition, stems from the fact that any solution of
$\SDE(\{V_i\}_{i=1}^{\ell}, \bfB^{1:\ell}, \bfX_0)$, where $\bfX_0$ is a
$\M$-valued random variable and $\bfB^{1:\ell}$ a $\rset^\ell$-valued Brownian
motion is a diffusion process with generator $\generator$ such that for any
$f \in \rmc^\infty(\M)$, $\generator(f) = \sum_{i=1}^\ell V_i(V_i(f))$. The
\emph{extrinsic} approach is particularly convenient since the SDE appearing in 
\cref{prop:extrinsic_brownian} can be seen as an SDE on the Euclidean space
$\rset^p$. 

We finish this paragraph, by investigating the behavior of the Brownian motion
in local coordinates. For simplicity, we assume here that we have access to a
system of global coordinates. In the case where the coordinates are strictly
local then we refer to \citet[Chapter 5, Theorem 1]{ikeda1989sto} for a
construction of a global solution by patching local solutions. We denote
$\{X_k, X_{i,j}\}_{1 \leq i,j,k \leq d}$ such that for any $u \in \FM$,
$\{X_k(u), X_{i,j}(u)\}_{1 \leq i,j,k \leq d}$ is a basis of $\mathrm{T}_u \FM$,
similarly as in the previous section. Using \cref{eq:horizontal_lift} we get
that $(\bfU_t)_{t \geq 0} = (\{\bfX^k_t, \bfE_t^{i,j}\}_{1 \leq i,j,k \leq d})$
obtained in \cref{prop:intrinsic_brownian} is given in the global coordinates for
any $i,j,k \in \{1, \dots, d\}$ by
\begin{equation}
  \textstyle{
    \rmd \bfX_t^k = \sum_{j=1}^d \bfE_t^{k,j} \circ \rmd \bfB_t^k  , \qquad \rmd \bfE_t^{i,j} = - \sum_{n=1}^d \{\sum_{\ell, m=1}^d \bfE_t^{\ell,n}\bfE_t^{m,j} \Gamma_{\ell,m}^{i}(\bfX_t)\} \circ \rmd \bfB_t^n  . 
    }
  \end{equation}
  By definition of the Stratanovitch integral we have that for any $k \in \{1, \dots, d\}$
  \begin{equation}
    \textstyle{
      \rmd \bfX_t^k = \sum_{j=1}^d \{ \bfE_t^{k,j} \rmd \bfB_t^k +(1/2) \rmd [\bfE_t^{k,j}, \bfB_t^j]_t \}  .
      }
    \end{equation}
    Let $(\bfM_t)_{t \geq 0} = (\{\bfM_t^k\}_{k=1}^d)_{t \geq 0}$ such that for
    any $t \geq 0$ and $k \in \{1, \dots, d\}$
    $\bfM_t^k = \sum_{j=1}^d \int_0^t \bfE_t^{k,j} \rmd \bfB_t^k$. We obtain
    that $\rmd \bfM_t = G(\bfX_t)^{-1/2} \rmd \bfB_t$ for some $d$-dimensional
    Brownian motion $(\bfB_t)_{t \geq 0}$, using L\'evy's characterization of
    Brownian motion. In addition, we have that for any
    $k, j \in \{1, \dots, d\}$
    \begin{equation}
      \textstyle{[\bfE^{k,j}, \bfB^j]_t = -\sum_{\ell, m=1}^d \int_0^t \bfE_t^{\ell, j} \bfE_t^{m,j} \Gamma_{\ell, m}^k(\bfX_t) \rmd t }
    \end{equation}
    Hence, using this result and the fact that
    $\sum_{j=1}^d \bfE_t^{\ell, j} \bfE_t^{m,j} = g^{\ell,m}(\bfX_t)$, we get
    that for any $k \in \{1, \dots, d\}$
    \begin{equation}
      \textstyle{\rmd \bfX_t^k =-  (1/2) \sum_{\ell, m=1}^d g^{\ell,m}(\bfX_t) \Gamma_{\ell, m}^k(\bfX_t) \rmd t + (G(\bfX_t)^{-1/2} \rmd \bfB_t)^k  . }
    \end{equation}
    Note that this result could also have been obtained using the expression of
    the Laplace-Beltrami in local coordinates.


    \paragraph{Brownian motion and random walks}

    In the previous paragraph we consider three SDEs to obtain a Brownian motion
    on $\M$ (stochastic development, isometric embedding and local
    coordinates). In this section, we summarize results from
    \cite{jorgensen1975central} establishing the limiting behavior of Geodesic
    Random Walks (GRWs) when the stepsize of the random walk goes to $0$. This will be
    of particular interest when considering the time-reversal process. We start
    by defining the geodesic random walk on $\M$, following \citet[Section
    2]{jorgensen1975central}.

    Let $\{ \nu_x \}_{x \in \M}$ such that for any $x \in \M$,
    $\nu_x: \mcb{\mathrm{T}_x \M} \to \ccint{0,1}$ with
    $\nu_x(\mathrm{T}_x \M) =1$, \ie \ for any $x \in \M$, $\nu_x$ is a
    probability measure on $\mathrm{T}_x \M$. Assume that for any $x \in \M$,
    $\int_{\M} \normLigne{v}^3 \rmd \nu_x(v)< +\infty$. In addition assume that
    there exists $\mu^{(1)} \in \XM$ and $\mu^{(2)} \in \XMdeux$, where
    $\XMdeux$ is the section
    $\Gamma(\M, \sqcup_{x \in \M} \mathcal{L}(\mathrm{T}_x \M))$, such that for
    any $x \in \M$, $\int_{\M} v \rmd \nu_x(v) = \mu^{(1)}(x)$ and
    $\int_{\M} v \otimes v \rmd \nu_x(v) = \mu^{(2)}(x)$. In addition, we assume
    that for any $x \in \M$,
    $\Sigma(x) = \mu^{(2)}(x) - \mu^{(1)}(x) \otimes \mu^{(1)}(x)$ is strictly
    positive definite and that there exists $\Ltt \geq$ such that for any
    $x, y \in \M$, $\tvnorm{\nu_x - \nu_y} \leq \Ltt d(x,y)$. Where we have that
    for any $\nu_1 \in \Pens(\mathrm{T}_x \M)$ and $\nu_2 \in \Pens(\mathrm{T}_y \M)$,
    \begin{equation}
      \tvnorm{\nu_x - \nu_y} = \sup \ensembleLigne{\nu_1[f] - \Gamma_{x}^y(\gamma)_\# \nu_2[f]}{\gamma \in \mathrm{Geo}_{x,y}, \ f \in \rmc(\mathrm{T}_x \M)}  . 
    \end{equation}
    Note that if $d(x,y) \leq \vareps$ then for some $\vareps > 0$ we have that $\abs{\mathrm{Geo}_{x,y}}=1$.


    \begin{definition}[Geodesic random walk]
      Let $X_0$ be a $\M$-valued random variable.  For any $\gamma > 0$, we
      define $(\bfX_t^{\gamma})_{t \geq 0}$ such that $\bfX_0^\gamma = X_0$ and
      for any $n \in \nset$ and $t \in \ccint{0, \gamma}$,
      $\bfX_{n\gamma + t} = \exp_{\bfX_{n \gamma}}[t\gamma \{ \mu_n +
      (1/\sqrt{\gamma}) (V_n - \mu_n)\}]$, where $(V_n)_{n \in \nset}$ is a sequence
      of random variables in such that for any $n \in \nset$, $V_n$
      has distribution $\nu_{\bfX_{n \gamma}}$ conditionally to $\bfX_{n \gamma}$.
    \end{definition}

    For any $\gamma > 0$, the process
    $(X_n^\gamma)_{n \in \nset} = (\bfX_{n \gamma}^\gamma)_{n \in \nset}$ is
    called a geodesic random walk. In particular, for any $\gamma>0$ we denote
    $(\Rker_n^{\gamma})_{n \in \nset}$ the sequence of Markov kernels such that
    for any $n \in \nset$, $x \in \M$ and $\msa \in \mcb{\M}$ we have that
    $\updelta_x \Rker(\msa) = \Pbb(X_n^\gamma \in \msa)$, with $X_0^\gamma =
    x$. The following theorem establishes that the limiting dynamics of a
    geodesic random walk is associated with a diffusion process on $\M$ whose
    coefficients only depends on the properties of $\nu$ \cite[see][Theorem
    2.1]{jorgensen1975central}.

    \begin{theorem}[Convergence of geodesic random walks]
      \label{thm:jorgensen_appendix}
      For any $t \geq 0$, $f \in \rmc(\M)$ and $x \in \M$ we have that
      $\lim_{\gamma \to 0} \normLigne{ \Rker_{\gamma}^{\ceil{t/\gamma}}[f] -
        \Pker_t[f]}_{\infty} = 0$, where $(\Pker_t)_{t \geq 0}$ is the
      semi-group associated with the infinitesimal generator
      $\generator: \ \rmc^\infty(\M) \to \rmc^\infty(\M)$ given for any
      $f \in \rmc^\infty(\M)$ by
      $\generator(f) = \langle p_{\textup{ref}}^1, \nabla f \rangle_{\M} + (1/2) \langle
      \Sigma, \nabla^2f \rangle_{\M}$.
    \end{theorem}   

    In particular if $\mu^{(1)} = 0$ and $\mu^{(2)} = \Id$ then the random walk
    converges towards a Brownian motion on $\M$ in the sense of the convergence
    of semi-groups. For any $x \in \M$ in local coordinates we have that
    $\Phi_\# \nu_x$ has zero mean and covariance matrix $G(x)$, where $\Phi$ is
    a local chart around $x$ and $G(x) = (g_{i,j}(x))_{1 \leq i,j \leq d}$ the
    coordinates of the metric in that chart.

    
\paragraph{Convergence of Brownian motion}

We finish this section with a few considerations regarding the convergence of
the Brownian motion on $\M$. Since we have assumed that $\M$ is compact we have
that there exist $(\Phi_k)_{k \in \nset}$ an orthonormal basis of $\Delta_\M$ in
$\mathrm{L}^2(p_{\textup{ref}})$, $(\lambda_k)_{k \in \nset}$ such that for any
$i, j \in \nset$, $i \leq j$, $\lambda_i \leq \lambda_j$ and $\lambda_0 = 0$, $\Phi_0=1$ and
for any $k \in \nset$, $\Delta_\M \Phi_k = -\lambda_k \Phi_k$. For any $t \geq 0$
and $x,y \in \M$,
$p_t(x,y) = \sum_{k \in \nset} \exp[-\lambda_k t] \Phi_k(x) \Phi_k(y)$ where for
any $f \in \rmc^\infty$ we have
\begin{equation}
  \textstyle{\expeLigne{f(\bfB_t^{\M,x})} = \int_\M p_t(x,y) f(y) \rmd p_{\textup{ref}}(y)  , }
\end{equation}
where $(\bfB_t^{\M,x})_{t \geq 0}$ is the Brownian motion on $\M$ with $\bfB_0^{\M,x} = x$
and $p_{\textup{ref}}$ is the probability measure associated with the Haussdorff measure on
$\M$. we also have the following result \cite[see][Proposition
2.6]{urakawa2006convergence}.

\begin{proposition}[Concergence of Brownian motion]
\label{prop:brownian_conv_repeat}
  For any $t > 0$, $\Pker_t$ admits a density $p_t$ w.r.t $p_{\textup{ref}}$ and
  $p_{\textup{ref}} \Pker_t = p_{\textup{ref}}$, \ie \ $p_{\textup{ref}}$ is an invariant measure for
  $(\Pker_t)_{t \geq 0}$. In addition, if there exists $C, \alpha \geq 0$ such
  that for any $t \in \ocint{0,1}$, $p_t(x,x) \leq C t^{-\alpha /2}$ then 
  for any $\pizero \in \Pens(\M)$ and for any $t \geq 1/2$ we have 
  \begin{equation}
    \textstyle{\tvnorm{\pizero \Pker_t - p_{\textup{ref}}} \leq C^{1/2} \rme^{\lambda_1 /2} \rme^{-\lambda_1 t}  ,}
  \end{equation}
  where $\lambda_1$ is the first non-negative eigenvalue of $-\Delta_\M$ in
  $\mathrm{L}^2(p_{\textup{ref}})$ and we recall that $(\Pker_t)_{t \geq 0}$ is the
  semi-group of the Brownian motion.
\end{proposition}
A review on lower bounds on the first positive eigenvalue
of the Laplace-Beltrami operator can be found in \citep{he2013lower}. These lower
bounds usually depend on the Ricci curvature of the manifold or its diameter. We
conclude this section by noting that in the non-compact case \citep{li1986large}
establishes similar estimates in the case of a manifold with non-negative Ricci
curvature and maximal volume growth.


%%% Local Variables:
%%% mode: latex
%%% TeX-master: "main"
%%% End:
