\section{Difference between ODE and SDE likelihood computations}
\label{sec:diff-betw-ode}

In this section, we show that the likelihood computation from
\cite{song2020score} does not coincide with the likelihood computation
obtained with the SDE model. We present our findings in the Riemannian setting
but our conclusions can be adapted to the Euclidean setting with arbitrary
forward dynamics. Recall that we consider a Brownian motion on the manifold as a forward process
$(\bfB_t^\M)_{t \in \ccint{0,T}}$ with $\{p_t\}_{t=0}^T$ the associated family
of densities. We have that for any $t \in \ccint{0,T}$ and $x \in \M$
\begin{equation}
  \label{eq:forward}
  \partial_t p_t(x) = \tfrac{1}{2} \Delta p_t(x) = \dive(\tfrac{1}{2}p_t \nabla \log p_t )(x)  . 
\end{equation}


\paragraph{ODE model}
In the case of the ODE model we define $(\bfX_t)_{t \in \ccint{0,T}}$ such that
$\bfX_0$ has distribution $\pi$ and satisfies
$\rmd \bfX_t = \tfrac{1}{2}  \nabla \log p_t(\bfX_t) \rmd t$. Note that the family of
densities $\{q_t\}_{t=0}^T$ associated with $(\bfX_t)_{t \in \ccint{0,T}}$ also
satisfies \cref{eq:forward}. Now, we consider
$(\bfhX_t)_{t \in \ccint{0,T}} = (\bfX_{T-t})_{t \in \ccint{0,T}}$ and note that it satisfies
\begin{equation}
  \label{eq:backward_flow_appendix}
 \rmd \bfhX_t = -\tfrac{1}{2}  \nabla \log p_{T-t}(\bfhX_t) \rmd t  .
\end{equation}
Finally, we consider $(\bfY_t^{\mathrm{ODE}})_{t \in \ccint{0,T}}$ which also satisfies
\cref{eq:backward_flow_appendix} and such that the distribution of $\bfY_0^{\mathrm{ODE}}$ is
$\piinv$. Denoting $\{q_t^{\mathrm{ODE}}\}_{t=0}^T$ the densities of
$(\bfY_t^{\mathrm{ODE}})_{t \in \ccint{0,T}}$ w.r.t. $\piinv$ we have for any $t \in \ccint{0,T}$ and $x \in \M$
\begin{equation}
  \label{eq:proba_flow_ode}
 \partial_t q_t^{\mathrm{ODE}}(x) =  \dive(q_t^{\mathrm{ODE}} -\tfrac{1}{2} \nabla\log p_{T-t} )(x)  . 
\end{equation}

\paragraph{SDE model}
When sampling we consider a process $(\bfY^{\mathrm{SDE}}_t)_{t \in \ccint{0,T}}$ such that
$\bfY^{\mathrm{SDE}}_0$ has distribution $\piinv$ and whose family of densities
$\{q_t^{\mathrm{SDE}}\}_{t=0}^T$ satisfies for any $t \in \ccint{0,T}$ and $x \in \M$
\begin{equation}
  \label{eq:proba_flow_sde}
  \partial_t q_t^{\mathrm{SDE}}(x) = -\dive(\log p_{T-t} q_t^{\mathrm{SDE}}(x)) +\tfrac{1}{2}\Delta q_t^{\mathrm{SDE}}(x) = \dive(q_t^{\mathrm{SDE}}\{-\nabla\log p_{T-t} + \tfrac{1}{2}\nabla\log q_t^{\mathrm{SDE}}\})(x)  . 
\end{equation}
Hence, \cref{eq:proba_flow_ode} and \cref{eq:proba_flow_sde} do not agree,
except if $q_t^{\mathrm{SDE}} = q_t^{\mathrm{ODE}} = p_{T-t}$ which is the case if and only if $\bfY^{\mathrm{SDE}}_0$ and
$\bfY_0^{\mathrm{ODE}}$ have the same distribution as $\bfX_T$. Note that it is possible to
evaluate the likelihood of the SDE model using that
\begin{equation}
  \partial_t \log q_t^{\mathrm{SDE}}(\bfY^{\mathrm{SDE}}_t) = \dive(-\nabla\log p_{T-t}(\bfY^{\mathrm{SDE}}_t) +\tfrac{1}{2}\nabla\log q_t^{\mathrm{SDE}}(\bfY^{\mathrm{SDE}}_t)) \rmd t  . 
\end{equation}
We can use the score approximation $\bm{s}_\theta(t,x)$ to approximate
$\nabla \log p_t(x)$ for any $t \in \ccint{0,T}$ and $x \in \M$. In order to
approximate $\nabla \log q_t^{\mathrm{SDE}}$, one can consider another neural network
$\bm{t}_\theta(t,x)$ approximating $\nabla \log q_t^{\mathrm{SDE}}(x)$ for any $t \in \ccint{0,T}$
and $x \in \M$. This approximation can be obtained using the implicit score loss
presented in \Cref{sec:riem-score-appr}.


%%% Local Variables:
%%% mode: latex
%%% TeX-master: "main"
%%% End:
