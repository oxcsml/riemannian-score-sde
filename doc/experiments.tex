\section{Experiments}
\label{sec:experiments}

We evaluate the model on a collection of datasets, each containing an empirical distribution of occurrences of earth and climate science events on the surface of the earth. These events are: volcanic eruptions \cite{volcanoe_dataset}, earthquakes \cite{earthquake_dataset}, floods \citep{flood_dataset} and wild fires \citep{fire_dataset}. In each case the earth is approximated as a perfect sphere. We compare to previous baseline methods: Riemannian Continuous Normalizing Flows \citep{mathieu2020riemannian}, Moser Flows \citep{rozen2021moser} and a mixture of Kent distributions \citep{peel2001fitting}. The mixture of Kent distributions is optimised using an EM algorithm and the optimal number of components is selected on a validation set.
Additionally, we consider another score-based generative model: a standard SBGM on the 2D place followed by the inverse stereographic projection which induces a density on the sphere \citep{gemici2016normalizing}.
More experimental details can be found in \cref{sec:exp_detail}.
We observe from \cref{tab:geoscience}, that the RSBGM model outperforms all other methods in density estimation, in particular by a large margin on the volcanic eruptions dataset.
% Qualitatively, we see on \cref{fig:geoscience} that 

\begin{table}[h]
    \centering
    \begin{tabular}{lrrrrr}
    % \toprule
     & \textbf{Volcano} & \textbf{Earthquake} & \textbf{Flood} & \textbf{Fire} \\
    \midrule
    Mixture of Kent & $-0.95_{\pm 0.14}$ & $0.14_{\pm 0.13}$ & $0.73_{\pm 0.07}$ & $-1.18_{\pm 0.06}$ \\
    Riemannian CNF & $-0.97_{\pm 0.15}$ & $0.19_{\pm0.04}$ & $0.90_{\pm0.03}$ & $-0.66_{\pm0.05}$ \\
    Moser Flow & $-2.02_{\pm 0.42}$ & $-0.09_{\pm0.02}$ & $0.62_{\pm 0.04}$ & $-1.03_{\pm 0.03}$ \\
    Stereographic Score-Based & ${-4.37}_{\pm ???}$ & ${-0.05}_{\pm ???}$ & ${1.32}_{\pm ???}$ & $0.11_{\pm ???}$ \\
    Riemannian Score-Based & $\bm{-5.56}_{\pm0.26}$ & $\bm{-0.21}_{\pm0.03}$ & $\bm{0.52}_{\pm0.02}$ & $\bm{-1.24}_{\pm 0.07}$\\
    \midrule 
    Dataset size & 827 & 6120 & 4875 & 12809 \\
    \bottomrule
    \end{tabular}
    \caption{
    Negative log-likelihood scores for each method on the earth and climate science datasets.
    Bold indicates statistically significant best method.
    Means and standard deviations are computed over 5 different runs.
    }
    \label{tab:geoscience}
\end{table}

\begin{figure}[t]
% \vspace{-0.8em}
  \centering
\begin{subfigure}{.33\textwidth}
  \includegraphics[width=\linewidth]{{pdf_earthquake_rsbgm}.png}
  \put(-150,40){\rotatebox{90}{Stereographic}}
\end{subfigure}\hfil
\begin{subfigure}{.33\textwidth}
  \includegraphics[width=\linewidth]{{pdf_earthquake_rsbgm}.png}
\end{subfigure}\hfil
\begin{subfigure}{.33\textwidth}
  \includegraphics[width=\linewidth]{{pdf_earthquake_rsbgm}.png}
\end{subfigure}\hfil
\begin{subfigure}{.33\textwidth}
  \includegraphics[width=\linewidth]{{pdf_earthquake_rsbgm}.png}
  \put(-150,40){\rotatebox{90}{Riemannian}}
  \put(-90,-10){Earthquake}
\end{subfigure}\hfil
\begin{subfigure}{.33\textwidth}
  \includegraphics[width=\linewidth]{{pdf_earthquake_rsbgm}.png}
  \put(-80,-10){Flood}
\end{subfigure}\hfil
\begin{subfigure}{.33\textwidth}
  \includegraphics[width=\linewidth]{{pdf_earthquake_rsbgm}.png}
  \put(-70,-10){Fire}
\end{subfigure}
\caption{
    Trained score-based generative models on earth sciences data.
    The learned density is colored green-blue.
    Blue and red dots represent training and testing datapoints, respectively.
  }
  \label{fig:geoscience}
% \vspace{-1.0em}
\end{figure}